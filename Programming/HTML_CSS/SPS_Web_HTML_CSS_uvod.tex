\section{Úvod}
Velmi pěkný tutoriál s ukázkami naleznete na: \url{https://www.w3schools.com/html/default.asp} a \url{https://www.w3schools.com/css/default.asp}\\

\vspace{0.5cm}

HTML a CSS jsou nerozlučná dvojice při tvorbě webových stránek. Proto je také v tomto dokumentu budeme popisovat zároveň. \\

HTML i CSS umožňuje nepřeberné množství možností, které se navíc neustále vyvíjejí. Tento dokument má sloužit především pro základní orientaci.

\subsection{HTML}
HTML je zkratka pro HyperText Markup Language - tedy Hypertextový značkovací jazyk.\\
Hypertextový znamená, že kromě samotného obsahu našich stránek - tedy textů a obrázků, budeme do naších souborů zapisovat ještě něco navíc.\\
To, co bude navíc budou v názvu zmíněné značky. Těmito značkami budeme označovat části textu a rozdělovat ho tak na nadpisy, odstavce, bloky, ale také těmito značkami určujeme typ textu - tučný, kurzíva.\\
Také pomocí značek můžeme vytvořit \uv{skupiny} textu, které mají vypadat stejně, pomocí značek vytvoříme odkazy, tabulky.

\subsection{CSS}
CSS označuje Kaskádové styly. Je to jazyk, pomocí kterého budeme určovat grafický styl našich stránek.\\
Pomocí CSS budeme určovat velikost písma, zarovnání, ohraničení, barvu pozadí, \dots

\subsection{Co potřebujeme}
Jediné, co budeme k práci s HTML a CSS potřebovat je \textbf{prohlížeč} (Firefox, Chrome, \dots) a \textbf{textový editor} (SublimeText, Poznámkový blok, Gedit, Vim, \dots) \textbf{! NE !} Word. V seznamu textových editorů je uveden Poznámkový blok pro ilustraci, že budeme opravdu psát pouze text - doporučuji použít něaký lepší editor, který vám bude pomáhat tím, že \textbf{zvýrazní důležitá slova}, která budeme k textu připojovat, bude Vám ukazovat \textbf{závorky}, které k sobě patří apd.\\
Pro zobrazení našeho souboru \textbf{nepotřebujeme internet} - prohlížeč bude zobrazovat náš soubor, který je uložený v našem počítači (stejně jako samotný prohlížeč) a veškeré informace o tom, jak nám má prohlížeč soubor zobrazit mu napíšeme do tohoto souboru my sami.