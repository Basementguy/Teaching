\section{Struktura souborů}
\subsection{Tagy}
\paragraph{HTML}
soubory jsou textové soubory. Jednotlivé značky, kterými popisujeme obsah (které přidáváme k obsahu) našich stránek se nazávají \textbf{tagy}.\\
Tyto tagy jsou až na několik málo výjimek párové - tedy vždy máme jeden otvírací tag a k němu jeden zavírací. Zavírací tag se liší pouze přidáním dopředného lomítka /.\\

\vspace{1cm}
Dvojici \textbf{otvírací-zavírací} tag si můžeme představit jako krabičku. Jednu krabičku můžeme mít uvnitř jiné krabičky, V krabičkce může být i mnoho různých krabiček, které uvnitř sebe opět mohou mít další krabičky.\\
Jediné co není možné je, aby se tagy křížili - Pokud zapíšu dva otvírací tagy, musím nejprve uzavřit ten druhý a teprve poté první.\\
Některé tagy umožňují vyplnit \textbf{doplňující informace} (např. velikost obrázku, adresa odkazu). Těmto \uv{informacím navíc} říkáme \textbf{atributy}.
\paragraph{CSS}
soubory jsou také textové soubory, ve kterých zapíšeme, jak se má daný tag zobrazovat. Zapíšeme to jednoduše tak, že napíšeme název tagu a za něj do složených závorek jednotlivé vlastnosti. Každá vlastnost končí středníkem.\\
Pokud má hodnota jednotky (např. tloušťka ohraničení), píšeme je těsně k číselné hodnotě. U některých vlastností totiž můžeme nastavovat různé parametry (např. u ohraničení - tloušťka, barva, styl (plný, čárkovaný)) a tyto různé parametry jsou odděleny mezerou.
\vspace{1cm}

Grafické vlastnosti můžeme zapisovat také do vlastního html souboru pomocí tagu <style> v <head>, a dokonce i přímo pro každý konkrétní prvek na stránce, pomocí atribudu \uv{style}. Této možnosti není od věci někdy využít, ale budeme se snažit co nejvíce věcí zapsat do samostatného css souboru. Především je to přehlednější a dále nám to také umožní měnit grafické zobrazení změnou css souboru, bez zásahu do souboru s obsahem stránky. Pokud zapíšete atribut \uv{style} přímo ke konkrétnímu prvku nejde již tuto nastavenou vlastnost ničím změnit.

\subsection{HTML soubor}
HTML soubor musí mít určité náležitosti, aby prohlížeč věděl, jak ho má zobrazit.\\
Na úvod zapíšeme , že se jedná o html dokument, následuje tag <html>, ve kterém bude vše ostatní.\\
V tagu <html> jsou dva hlavní tagy <head> a <body>. Do <head> zapisujeme nastavení stránky (vybraný css soubor, nadpis v liště prohlížeče apd.). V <body> je samotný obsah naší stránky, který nám prohlížeč zobrazí.\\
\begin{minipage}[t]{.45\textwidth}
\begin{code}
\begin{minted}[linenos, escapeinside=||]{html}
<!DOCTYPE html> |\label{scl:html_doctype}|
<html> |\label{scl:html_html_o}|
 <head> |\label{scl:html_head_o}|
  <link rel="stylesheet" type="text/css" href="graficky_soubor.css"> |\label{scl:html_ref_css}|
 </head> |\label{scl:html_head_c}|
 <body> |\label{scl:html_body_o}|
  <!-- Pro přehlednost je dobré, odsazovat tagy, které jsou uvnitř jiných tagů -->
  <h1>Toto je nadpis</h1> |\label{scl:html_h}|
  <p>Toto je odstavec.</p> |\label{scl:html_p}|

 </body> |\label{scl:html_body_c}|
</html>  |\label{scl:html_html_c}|
\end{minted}

\captionof{listing}{HTML úvod}
\label{code:html_uvod}
\end{code}
\end{minipage}
\begin{minipage}[t]{.45\textwidth}
\begin{enumerate}
\vspace{-0.4cm}
\item[ř. \ref{scl:html_doctype}:] Na úvod souboru zapíšeme, že se jedná o html soubor.
\item[ř. \ref{scl:html_html_o}-\ref{scl:html_html_c}:] Vše je uzavřeno do <html> tagu.
\vspace{0.2cm}
\item[ř. \ref{scl:html_head_o}-\ref{scl:html_head_c}:] Do tagu <head> zapisujeme především nastavení stránky.
\item[ř. \ref{scl:html_ref_css}:] Označení souboru css, ze kterého chceme číst grafické nastavení.
\item[ř. \ref{scl:html_body_o}-\ref{scl:html_body_c}:] Samotný obsah stránky zapíšeme do tagu <body>
\item[ř. \ref{scl:html_h}:] Nadpis 
\item[ř. \ref{scl:html_p}:] Odstavec
\end{enumerate}
\end{minipage}\\

\begin{minipage}[t]{.45\textwidth}
\begin{code}
\begin{minted}[linenos, escapeinside=||]{css}
/*Všechny vlastnosti jsou ukončené středníkem!*/
body { |\label{scl:html_css_body}|
  background-color: lightblue; |\label{scl:html_css_body_bg}|
}

h1 { |\label{scl:html_css_h1}|
  color: navy; |\label{scl:html_css_h1_color}|
  margin-left: 20px; |\label{scl:html_css_h1_margin}|
}
\end{minted}

\captionof{listing}{CSS úvod}
\label{code:css_uvod}
\end{code}
\end{minipage}
\begin{minipage}[t]{.45\textwidth}
\begin{enumerate}
\vspace{0.4cm}
\item[ř. \ref{scl:html_css_body}. \ref{scl:html_css_h1}:] Zapíšeme název tagu a do složených závorek všechny požadované vlastnosti.
\item[ř. \ref{scl:html_css_body_bg}:] Nastavení barvy pozadí celé stránky.
\item[ř. \ref{scl:html_css_h1_color}:] Nastavení barvy nadpisů typu h1.
\item[ř. \ref{scl:html_css_h1_margin}:] Nastavení odsazení nadpisů h1.\\Jednotky píšeme těsně k hodnotám
\end{enumerate}
\end{minipage}

\subsubsection{Metadata}
Je vhodné do naší stránky zapsat, kdo je jejím autorem, kdy proběhla poslední aktualizace apd..\\
To provedeme nejlépe pomocí tagu <meta>.\\
Také pomocí <meta> můžeme (a měli bychom) nastavit, jak má prohlížeč zobrazit stránku (např. v jakém přiblížení).\\

\begin{minipage}[t]{.45\textwidth}
\begin{code}
\begin{minted}[linenos, escapeinside=||]{html}
<!DOCTYPE html> 
<html> 
 <head> 
  <meta name="Autor" content="Nejlepší jméno"> |\label{scl:html_meta_autor}|
  <meta name="Poslední úprava" content="teď"> |\label{scl:html_meta_uprava}|
  <meta name="viewport" content="width=device-width, initial-scale=1.0"> |\label{scl:html_meta_viewport}|
 </head>
 <body>
 </body> 
</html>  
\end{minted}

\captionof{listing}{Metadata}
\label{code:html_meta}
\end{code}
\end{minipage}
\begin{minipage}[t]{.45\textwidth}
\begin{enumerate}
\item[ř. \ref{scl:html_meta_autor}, \ref{scl:html_meta_uprava}:] Nastavení metadat o stránce.
\item[ř. \ref{scl:html_meta_viewport}:] Nastavení uvodního viewportu - jak se má stránka zobrazit

\end{enumerate}
\end{minipage}\\


\subsubsection{Komentáře}
Psát si ke kódu komentáře je velmi vhodné. Komentář je text, který se na výsledené podobě stránky nijak neprojeví a zapisujeme do něj vysvětlivky ke kódu.\\
Třeba pro někoho, kdo náš kód bude spravovat, ale i sami prosebe, až se ke kódu vrátíme po měsíci a nebudeme si pamatovat, co jsme to tu vlastně psali). Můžeme také pomocí komentářú zřetelně oddělit některé části stránky (třeba tabulku, kterou často upravujeme) a zpříjemnit si tak orientaci v kódu.\\
\paragraph{v HTML} souboru zapisujeme komentáře to tagu $ <!-- \dots --> $
\paragraph{v CSS} souboru zapisujeme komentáře mezi $ /* \dots */ $
 