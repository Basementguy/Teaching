\section{Základní přehled tagů}
Tagů a jejich grafických úprav můžeme využít nepřeberné množství. Proto vždy, když budete chctít něco vytvořit, podívejte se do dokumentace a téměř jistě najdete jednoduchou možnost, jak to provést.\\
Zde si ukážeme minimální souhrn tagů, které \uv{musíte} znát. Samozřejme i tyto tagy naleznete v dokumentaci, ale na druhou stranu hledat něco každých 5 minut značně zdržuje.\\
\vspace{1cm}
Vzhledem k tomu, že základní grafická úprava pomocí css je pro velkou část tagů stejná. Nebude zde vypsána u každého tagu, ale pouze u těch, kde se v něčem liší, je v něčem zajímavá.
\subsection{css}
V css nejčasteji využijeme úpravu:
\begin{itemize}
\item[barva textu - ] color: zapisujeme barvu slovy (anglicky), připadně číslem \url{https://www.w3schools.com/css/css_colors.asp}
\item[font - ] font: nastavujeme velikost, styl, tloušťku, rodinu  - \url{https://www.w3schools.com/css/css_font.asp}
\item[ohraničení - ] border: nastavujeme barvu, tloušťku, styl, zakulacení - \url{https://www.w3schools.com/css/css_border.asp}
\item[odkazy - ] můžeme nastavit různý styl pro odkaz, který je aktivní (míří na něj myš), je již navštívený, právě jsme na něj klikli - \url{https://www.w3schools.com/css/css_link.asp} 
\item[tabulky - ] ohromné množství možností - nastavujeme zvlášť vlastnosti celé tabulky, řádků a buněk - \url{https://www.w3schools.com/css/css_table.asp}
\item[pozadí - ] můžeme nastavit, barvu, obrázek, každý prvek může mít své pozadí - \url{https://www.w3schools.com/css/css_background.asp}
\item[zarovnání - ] align: zarovnání vlevo, vpravo, na střed, nahoru, dolů - \url{https://www.w3schools.com/css/css_align.asp}
\item[odsazení - ] máme dva typy odsazení - margin a padding - jaký je mezi nimi rozdíl ukazuje obrázek zde - \url{https://www.w3schools.com/css/css_boxmodel.asp}
\end{itemize}

\subsubsection{Jednotky}
Při zadávání rozměrů samozřejmě musíme zadat i jednotky.\\
Jednotky existují \textbf{absolutní}, které udávají pevnou velikost - mm, cm, in, px (milimetry, centimetry, inche, pixely)\\
a \textbf{relativní}, které udávají velikost prvku vzhledem k jinému prvku (třeba velikosti obrazovky) - em, vw, vh, \% (relativně k font-size, setině šířky obrazovky, setině výšky obrazovky, k nadřazenému prvku (tomu, ve kterém je tento prvek))\\
\vspace{0.5cm}
Vybrat správné jednotky je velmi důležité, jinak bude stránka na některých zařízení (velká obrazovka, malá obrazovka) nečitelná.

\subsection{Nadpis - <h1>, <h2>, <h3>}
Do tagů <h1>, <h2>, <h3> (headings) zapisujeme nadpisy. Tedy text, který prohlížeč zobrazuje oddělený od dalšího textu a graficky výraznější.\\
Čísla za \uv{h} rozlišují úrovně nadpisů - <h1> je největší.

\subsection{Odstavec - <p>}
Do odstavce <p> (paragraph) píšeme prostý text.

\subsection{Odkaz - <a>}
Do tagu <a> vládáme prvek, který nás po kliknutí přesměruje na zadanou stránku (nebo její část). Jako odkaz můžeme použít v zásadě cokoliv - text, nadpis, obrázek, blok, \dots \\
Aby prohlížeč věděl, kterou stránku má zobrazit po kliknutí na odkaz, musíme vyplnit atribut \uv{href}.

\begin{minipage}[t]{.45\textwidth}
\begin{code}
\begin{minted}[linenos, escapeinside=||]{html}
<!DOCTYPE html> 
<html> 
 <head> 
  <link rel="stylesheet" type="text/css" href="graficky_soubor.css"> 
 </head> 
 <body> 
 
  <a href="www.wikipedia.cz">Odkaz na wiki.</a> |\label{scl:html_a_wiki}|
  <a href="moje_druha_stranka.html">Odkaz na naši další stránku.</a> |\label{scl:html_a_druha}|

 </body>
</html> 
\end{minted}

\captionof{listing}{Odkaz}
\label{code:html_odkaz}
\end{code}
\end{minipage}
\begin{minipage}[t]{.45\textwidth}
\begin{enumerate}
\vspace{-0.4cm}
\item[ř. \ref{scl:html_a_wiki}:] Odkaz na internetovou stránku.
\item[ř. \ref{scl:html_a_druha}:] Odkaz na náš jiný soubor.
\end{enumerate}
\end{minipage}\\

\subsection{Obrázek - <img>}
Obrázek můžeme ve stránce zobrazovat pomocí tagu <img> (image). Tento tag je pouze otevírací. Nutný atribut je \uv{src}, kterým určíme kde je obrázek uložen (může být uložen v počítači, nebo na internetové adrese). Vhodné atributy jsou také \uv{alt}, kterým nastavíme náhradní text (zobrazíse, pokud se nepodaří načíst obrázek) a nezapomenout nastavit výšku a šířku - pomocí atributu \uv{style} (což jsou css, jen vložené přímo k jednomu konkrétnímu prvku).

\subsection{Seznam - <ol>, <ul>}
Seznamů máme dva typy - Číslovaný <ol> (ordered list) a Nečíslovaný <ul> (unordered list).\\
Číslovaný seznam může být číslovaný nejen čísly, ale i písmeny, řeckými číslicemi.\\
Nečíslovaný seznam může mít různé typy odrážek - kruh, kružnice, \dots \\
Celý seznam  píšeme mezi tagy <ol>, nebo <ul>. Jednotlivé \textbf{prvky seznamu} pak do tagů \textbf{<li>} (list item).\\
Ukázky seznamu zde \url{https://www.w3schools.com/html/html_lists.asp}

\subsection{Tabulka - <table>}
Každá tabulka je vložena do tagu <table>.\\
Dále musí prohlížeč vědět, co jsou jednotlivé řádky a buňky tabulky. Ty zapíšeme do tagů <tr> (table row) pro každý řádek. A do řádku zapíšeme tagy <td> (table data) pro každou buňku v řádku.\\
Dále často chceme odlišit první řádek - hlavičku tabulky s nadpisem sloupců. To provedeme pomocí použití <th> (table head) tam, kde bychom jinak použili <td>.\\

\vspace{0.5cm}
Pro lepší možnosti grafického zobrazení tabulky je vhodné ji rozdělit na záhlaví - tělo - zápatí, pomocí <thead> - <tbody> - <tfoot>\\
Také můžeme tabulce přidat nadpis pomocí <caption>\\

\vspace{1cm}
U tabulek - a především jejich grafického nastavení je nejlepší si rovnou prohlédnout výsledek a vybrat si, co se Vám do vašich stránek líbí:
\begin{itemize}
\item[html:] \url{https://www.w3schools.com/html/html_tables.asp} 
\item[css:] \url{https://www.w3schools.com/css/css_table.asp}
\end{itemize}


\subsection{Blok - <div>}
Blok <div> slouží k uzavření několika (nebo jen jednoho) prvků do jednoho bloku a jejich společné formátování. Lze tak například nastavit pozadí polovině stránky - polovinu obsahu uzavřeme do bloku a tomuto bloku nastavíme požadované pozadí.

\subsection{Jednořádkový blok - <span>}
Blok <span> používáme obdobně jako <div>. Zatímco <div> se ale vždy roztáhne co nevíce do šířky, blok <span> zůstane jen těsně okolo vnitřního prvku.

\begin{minipage}[t]{.45\textwidth}
\begin{code}
\begin{minted}[linenos, escapeinside=||]{html}
<!DOCTYPE html> 
<html> 
 <head> 
  <link rel="stylesheet" type="text/css" href="graficky_soubor.css"> 
 </head> 
 <body> 
  <!-- Zkuste v prohlížečí přiblížit a oddálit stránku -->
  <span style="display: inline-block; border:solid green;"> |\label{scl:html_span}|
    <img src="https://i.kym-cdn.com/entries/icons/original/000/000/063/Rage.jpg">
  </span>
  <br>
  <div style="border:solid green;"> |\label{scl:html_div}|
   <img src="https://i.kym-cdn.com/entries/icons/original/000/000/063/Rage.jpg">
  </div>
  <br>
 </body>
</html>
\end{minted}

\captionof{listing}{Odkaz}
\label{code:html_odkaz}
\end{code}
\end{minipage}
\begin{minipage}[t]{.45\textwidth}
\begin{enumerate}
\vspace{-0.4cm}
\item[ř. \ref{scl:html_span}:] Zobrazí ohraničení jen okolo obrázku
\item[ř. \ref{scl:html_div}:] Zobrazí ohraničení až k okrajům stránky
\end{enumerate}
\end{minipage}\\

\subsection{Identifikátor - id}
Identifikátor - id je \textbf{atribut}. Tedy ho přiřazujeme nějakému prvku - tagu.\\
Můžeme si takto označit konkrétní prvek a pak se na něj odkazovat. Jedno id může mít pouze jeden prvek na stránce. Tedy musí být unikátní.\\
Můžeme vytvořit odkaz tak, že prohlížeč zobrazí tento prvek jako první na stránce.\\
V CSS můžeme upravit vlastnosti pouze tohoto konrétního prvku

\subsection{Třída - class}
Třída - class je také \textbf{atribut} a tedy ji přiřazujeme nějakému tagu.\\
Funguje v zásadě stejně jako id, jen s tím rozdílem, že v jedné třídě může být zařazeno více prvků.\\

\begin{minipage}[t]{.45\textwidth}
\begin{code}
\begin{minted}[linenos, escapeinside=||]{html}
<!DOCTYPE html> 
<html> 
 <head> 
  <link rel="stylesheet" type="text/css" href="graficky_soubor.css"> 
 </head> 
 <body> 
  <p id="1" class="dulezite">Nápis ve třídě "dulezite" a s id "1"</p> |\label{scl:html_id_class}|
  <p class="dulezite">Nápis ve třídě "dulezite"</p> |\label{scl:html_class}|
  <p id="2">Nápis s id "2"</p> |\label{scl:html_id}|
 </body>
</html>
\end{minted}

\captionof{listing}{HTML - id, class}
\label{code:html_id_class}
\end{code}
\end{minipage}
\begin{minipage}[t]{.45\textwidth}
\begin{enumerate}
\vspace{-0.4cm}
\item[ř. \ref{scl:html_id_class}:] Prvek může být ve třídě (i v několika) a mít id
\item[ř. \ref{scl:html_class}:] V jedné třídě může být více prvků
\item[ř. \ref{scl:html_id}:] Další prvek musí mít jiné id - id se nemohou opakovat
\end{enumerate}
\end{minipage}\\

\begin{minipage}[t]{.45\textwidth}
\begin{code}
\begin{minted}[linenos, escapeinside=||]{html}
 #1 { |\label{scl:css_id_1}|
	font-size: 10px; 
 }
 
 #2 { |\label{scl:css_id_2}|
	font-size: 20px;
 }
 
 .dulezite { |\label{scl:css_class}|
	color: red; 
 }
\end{minted}

\captionof{listing}{CSS - id, class}
\label{code:html_id_class}
\end{code}
\end{minipage}
\begin{minipage}[t]{.45\textwidth}
\begin{enumerate}
\vspace{-0.4cm}
\item[ř. \ref{scl:css_id_1}, \ref{scl:css_id_2}:] Na id odkazujeme pomocí \#
\item[ř. \ref{scl:css_class}:] Na třídu odkazujeme pomocí . (tečky)
\end{enumerate}
\end{minipage}\\

\subsection{Rozložení - layout}
Jsou místa na stránce, která jsou tradičně využívána k zobrazení určitých informací.\\
Pro využítí schopnosti prohlížečů tato místa zobrazit uživateli přehledně je vhodné využít připravené layout prvky.\\
Nejlépe pochopíme o co jde pří dostatečně dlouhém dívání na obrázek (ne hned první ukázku, ale o trochu pod ní) na stránce \url{https://www.w3schools.com/html/html_layout.asp}






