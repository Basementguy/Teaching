\section{Session a cookies}
Často si chceme na našich stránkách udržet libovolnou informaci i při přechodu na jinou stránku (v rámci naší webové prezentace). Například se chceme příhlásit pouze jednou a rádi bychom, aby toto přihlášení vydrželo celou dobu, co na stránkách pracujeme. Případně si nastavíme barvu pozadí a chceme mít stejné pozadí po celou dobu.\\
Je zřejmé, že si takovou informaci musíme někam uložit. Musíme ji uložit na místo, kde zůstane uchována i při zavření stránky - a otevření jiné stránky, nebo i té samé.\\
Možnosti kam uložit informace budeme využívat dvě:
\begin{enumerate}
\item[\textbf{Na server}] pomocí \textbf{session} 
\item[\textbf{Do PC}] pomocí \textbf{cookies}
\end{enumerate}

\subsection{Session}
Pomocí session si můžeme uložit informace na server, který spouští naše php skripty.\\
Nejprve navážeme se serverem trvalé spojení (session) - toto spojení je platné (server pozná, že jsme novou stránku otevřeli opět my) do doby, dokud nezavřeme prohlížeč (velmi komplikovaným způsobem jde přerušit i pomocí php kódu). Při zavření pouze jedné karty, zůstane spojení navázané.\\
\vspace{0.5cm}
V průběhu spojení si informace ukládáme do asociativního pole \$\_SESSION .   

\begin{minipage}[t]{.45\textwidth}
\begin{code}
\begin{minted}[linenos, escapeinside=!!]{html+php}
<?php
session_start() !\label{scl:session_start} !
?>
<html>
<body>

<?php
$_SESSION["username"] = "MojeJmeno"; !\label{scl:session_save} !
$_SESSION["font_size"] = 50;
  
echo '<p style="font-size: $_SESSION[\"font_size\"];">'; !\label{scl:session_use} !
echo 'Přihlášen uživatel: $_SESSION["username"] </p>'; 

unset($_SESSION['username']); !\label{scl:unset} !
session_destroy(); !\label{scl:session_destroy} !
?>

</body>
</html> 
\end{minted}

\captionof{listing}{Session}
\label{code:session}
\end{code}
\end{minipage}
\begin{minipage}[t]{.45\textwidth}
\begin{enumerate}
\item[ř. \ref{scl:session_start}:] Navázání spojení (session) se serverem.\\Musí být první příkaz na stránce.
\item[ř. \ref{scl:session_save}:] Uložení informací.
\item[ř. \ref{scl:session_use}:] Použití informací.
\item[ř. \ref{scl:unset}:] Smazání jedné proměnné.
\item[ř. \ref{scl:session_destroy}:] Smazání všech dat.
\end{enumerate}
\end{minipage}

\subsection{Cookies}
Cookies jsou informace, které si stránka ukládá do PC učivatele.\\
Není zcela samozřejmé, že si stránka (která především slouží ke čtení) ukládá cokoliv na uživatelské zařízení - a zasahuje tak do něj (zabírá pamět). Proto s tímto zásahem do svého zařízení musí uživatel souhlasit - to je důvod, proč jste se již jistě setkali s tím, že jste na stránce museli odškrtnout souhlas s cookies.\\

Informace \textbf{ukládáme a měníme} pomocí funkce \uv{\textbf{setcookie()}}.\\
\textbf{Čteme} je z asociativního \textbf{pole \$\_COOKIE}.  

\begin{minipage}[t]{.45\textwidth}
\begin{code}
\begin{minted}[linenos, escapeinside=!!]{html+php}
<?php
setcookie("username", "MojeJmeno"); !\label{scl:cookie_set1} !
setcookie("font-size", 50); !\label{scl:cookie_set2} !

/*Tento řádek zkuste zakomentovat*/
setcookie("username", "", time() - 3600);  !\label{scl:cookie_delete} !
?>
<html>
<body>

<?php
echo '<p style="font-size: $_COOKIE[\"font_size\"];">'; !\label{scl:cookie_use} !
echo 'Přihlášen uživatel: $_COOKIE["username"] </p>'; 
?>

</body>
</html> 
\end{minted}

\captionof{listing}{Cookie}
\label{code:cookie}
\end{code}
\end{minipage}
\begin{minipage}[t]{.45\textwidth}
\vspace{-1cm}
\begin{enumerate}
\item[ř. \ref{scl:cookie_set1}, \ref{scl:cookie_set2}:] Vytvoření (uložení) cookie. Funkce \uv{setcookie()} se musí objevit před <html> tagem
\item[ř. \ref{scl:cookie_use}:] Použití informací.
\item[ř. \ref{scl:cookie_delete}:] Smazání jednoho z cookie. Nastavíme mu dobu platnosti na čas (v sekundách) v minulosti.\\
Zde jen ilustrativní přiklad - obvykle by tento řádek byl na jiné stránce.
\end{enumerate}
\end{minipage}

