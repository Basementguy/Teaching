\section{Řídící struktury}
Každý program běží postupně po řádcích od 1. řádku dále a vykonává příkazy přesně v tom pořadí, jak jdou za sebou.\\
Někdy (velmi často) ale chceme, aby se některé řádky (přikazy) přeskočily - neprovedly se. Někdy také chceme, aby se některé přikazy provedly opakovaně - vícekrát.\\
K tomuto slouží takzvané řídící struktury - speciální příkazy, které řídí, který řádek (příkaz) se provede jako další. Říkáme, že řídí běh programu.\\
To, že je nějaký řádek uvnitř řídící struktury (např. je to řádek, který chceme přeskočit) se v Pythonu pozná podle \textbf{odsazení}. Odsazení může být jakékoliv (mezera, tři mezery, ...) ale doporučuji tabulátor. Všechny řádky, které jsou ve stejné \uv{skupince} musí být odsazené stejně (např. všechny o jeden tabulátor;). 

\subsection{Odkazy}
Operátory\\ \url{https://www.w3schools.com/python/python_operators.asp} \\ \url{https://naucse.python.cz/course/pyladies/beginners/comparisons/}\\
Podmínka - Větvení\\ \url{https://www.w3schools.com/python/python_conditions.asp} \\ \url{https://naucse.python.cz/course/pyladies/beginners/comparisons/}\\
Cykly \\ \url{https://www.w3schools.com/python/python_while_loops.asp}\\ \url{https://www.w3schools.com/python/python_for_loops.asp} \\ \url{https://naucse.python.cz/course/pyladies/beginners/while/}

\subsection{Podmínka - Větvení}
Můžeme si představit, že program běží po jednotlivých větvých - proto tomu říkáme větvení.\\ 
Stejně jako u větví na stromě, se v nějakém místě programu můžeme vydat dvěma směry (po dvou různých větvích). Na rozdíl od větví na stromě se ale jednotlivé větve v programu mohou opět spojit do jedné. 

\subsubsection{if}
Nejjednodušší podmínka \uv{if} nám poslouží v případě, že chceme některé příkazy provést jen někdy - jen pokud je splněna daná podmínka. Pokud podmínka splněna není, přikazy se přeskočí.\\
\begin{minipage}[t]{.45\textwidth}
\begin{code}
\begin{minted}[linenos]{python}
trabant.jed()

if trabant.je_v_domecku():
	trabant.dopln_palivo()
	
trabant.jed()
\end{minted}

\captionof{listing}{if}
\label{code:if}
\end{code}
\end{minipage}
\begin{minipage}[t]{.45\textwidth}
ř. 1:	Provede se vždy\\
ř. 3:	Zjišťuje se, zda je trabant v domečku\\
ř. 4:	Provede se jen v připadě, že je trabant v domečku\\
Pokud v domečku není, řádek se přeskočí\\
ř. 6:	Provede se vždy
\end{minipage}\\

Za klíčové slovo \uv{if} píšeme podmínku - cokoliv, o čem umí Python rozhodnout, zda je to splněné, nebo ne - zda je to \textit{True}, nebo \textit{False}.\\
Můžeme zde přímo zapsat \textit{True}, nebo \textit{False}, často zde píšeme porovnání: < , > , == , != , <= , >= .\\
\begin{minipage}[t]{.45\textwidth}
\begin{code}
\begin{minted}[linenos]{python}
trabant.jed()

if trabant.get_rychlost() > 50:
	policista.dej_pokutu(trabant)
	
trabant.jed()
\end{minted}

\captionof{listing}{if porovnání}
\label{code:if_porovnani}
\end{code}
\end{minipage}
\begin{minipage}[t]{.45\textwidth}
ř. 1:	Provede se vždy
ř. 3:	Zjišťuje se rychlost trabantu a porovná se s hodnotou 50\\
ř. 4:	Provede se jen v připadě, že trabant jede rychleji, než 50\\
Pokud jede 50 a méně, řádek se přeskočí\\
ř. 6:	Provede se vždy
\end{minipage}\\

\subsubsection{if-else}
Často chceme aby se při splnění podmínky vykonaly některé příkazy a při nesplnění podmínky se vykonaly jiné. Chceme tedy pomocí podmínky vybrat jednu ze dvou skupin příkazů (řádků).\\
K tomu slouží konstrukce \uv{if-else}. Při splnění podmínky (True) se (stejně jako v obyčejném \uv{if}) provedou příkazy v části po \uv{if}. Při nesplnění podmínky (False) se provedou příkazy v části po \uv{else}. Jedna z těchto dvou skupin příkazů se tedy provede vždy.\\
\begin{minipage}[t]{.45\textwidth}
\begin{code}
\begin{minted}[linenos]{python}
if trabant.get_body() > ferrari.get_body():
	hra.set_vitez(trabant)
else:
	hra.set_vitez(ferrari)
\end{minted}

\captionof{listing}{if-else}
\label{code:if_else}
\end{code}
\end{minipage}
\begin{minipage}[t]{.45\textwidth}
\vspace{.5cm}
ř. 1:	Porovnání počtu bodů trabantu a ferrari\\
ř. 2:	Pokud je podmínka splněna == True, je vítěz trabant\\
ř. 4:	Pokud není podmínka splněna == False, je vítěz ferrari\\
Kontrolní otázka... Kdo je vítěz, pokud mají stejný počet bodů?\\
\end{minipage}\\

\subsubsection{elif}
Opravíme kód \ref{code:if_else}. Někdy potřebujeme vybrat ne mezi dvěma příkazy, ale mezi více. K tomu slouží konstrukce \uv{elif} - dovolí nám vložit další \uv{větev}. Takových větví může býl libovolné množství a \textbf{program se vydá} vždy jen tou z nich, u které je \textbf{podmínka splněna jako první} (kontrolováno od \textbf{shora dolů}). Pokud není splněna žádná z podmínek, vydá se program větví \uv{else}.
\begin{minipage}[t]{.45\textwidth}
\begin{code}
\begin{minted}[linenos]{python}
if trabant.get_body() > ferrari.get_body():
	hra.set_vitez(trabant)
elif ferrari.get_body() > trabant.get_body():
	hra.set_vitez(ferrari)
else:
	hra.set_vitez("remíza")
\end{minted}

\captionof{listing}{if-elif-else}
\label{code:if_elif_else}
\end{code}
\end{minipage}
\begin{minipage}[t]{.45\textwidth}
\vspace{1.5cm}
ř. 2,4,6:	Provede se jen jeden z těchto příkazů - poté, co jedna z podmínek projde (je splněna), se ostatní ani nekontrolují\\
\end{minipage}\\

\subsubsection{Vnoření}
Podmínky (stejně jako jiné řídící struktury) můžeme takzvaně \textbf{vnořovat} - tedy \textbf{vkládat jednu do druhé}.
\begin{minipage}[t]{.45\textwidth}
\begin{code}
\begin{minted}[linenos]{python}
if trabant.je_ve_meste():
	if trabant.get_rychlost() > 50:
		policista.dej_pokutu(trabant)
else:
	if trabant.get_rychlost() > 90:
		policista.dej_pokutu(trabant)
\end{minted}

\captionof{listing}{if-else vnoření}
\label{code:if_else_vnoreni}
\end{code}
\end{minipage}
\begin{minipage}[t]{.45\textwidth}
\vspace{2.5cm}
ř. 1:	Zjištění zda je trabant ve městě, nebo mimo město\\
ř. 2-3:	Pokud je ve měste porovnává se rychlost s 50\\
ř. 4:	Není ve městě\\
ř. 5-6:	Pokud není ve měste porovnává se rychlost s 90
\end{minipage}\\

\subsection{Cykly}
Velmi často chceme, aby program provedl nějaký výpočet vícekrát. Buď proto, že tímto opakováním získáme požadovaný výsledek, nebo chceme stejnou operaci provést s více \uv{objekty} (např. poslat zprávu všem kamarádům ze seznamu).\\
K tomuto opakování slouží takzvané cykly.

\subsubsection{While}
Cyklus \uv{while} opakuje příkazy, \textbf{dokud je splněna} zadaná \textbf{podmínka}. Nejprve zkontroluje, zda je podmínka splněna - pokud ano, \textbf{vykoná všechny zadané příkazy} a poté zkontroluje podmínku znovu - a tak stále dokola, dokud při kontrole podmínky nezjistí, že podmínka již splněna není. Ve chvíli, kdy podmínka splněná není (ať už hned napoprvé, nebo kdykoliv později), přeskočí všechny zadané příkazy a program pokračuje dále za tímto \uv{while} cyklem.
\begin{minipage}[t]{.45\textwidth}
\begin{code}
\begin{minted}[linenos]{python}
trabant.jed()

while trabant.chci_palivo():
	benzinka.vyber_korunky(trabant)
	benzinka.dopln_trochu_paliva(trabant)
	
trabant.jed()	
\end{minted}

\captionof{listing}{While}
\label{code:while}
\end{code}
\end{minipage}
\begin{minipage}[t]{.45\textwidth}
\vspace{2.3cm}
ř. 1:	Provede se jednou\\
ř. 3:	Kontrola, zda trabant chce palivo\\
ř. 4-5:	Pokud je podmínka splněna (chce palivo (True)) vykonají se všechny příkazy uvnitř while\\Poté se znovu zkontroluje podmínka\\
ř. 7:	Ve chvíli, kdy podmínka splněna není (nechce palivo (False)) pokračuje program dále za celým cyklem (za všemi řádky uvnitř cyklu)
\end{minipage}\\

\subsubsection{For}
Cyklus \uv{for} používáme v Pythonu k procházení libovolného seznamu (resp. čehokoliv, co se skládá z více prvků). Takovým seznamem může být mimo jiné:
\begin{enumerate}
\item Pole (v Pythonu List) - seznam libovolných prvků
\item Řetězec (String) - seznam znaků (písmeno, číslice, tečka, mezera, atd.)
\item Range - seznam čísel
\item Textový soubor - seznam řádků
\end{enumerate}
Při procházení seznamu pomocí \uv{for} máme uložen \textbf{jeden} prvek ze seznamu při každém průchodu (procházíme příkazy uvnitř \uv{for}). Při prvním průchodu máme uložen první prvek, při druhém průchodu druhý prvek atd.\\
Můžeme tak nějakou operaci (jeden nebo více příkazů) provést s každým prvkem seznamu zvlášť - a provést ji postupně se všemi prvky.\\
\begin{minipage}[t]{.45\textwidth}
\begin{code}
\begin{minted}[linenos]{python}
hra.zmen_rocni_obdobi(zima)

for auticko in hra.get_vsechna_auta():
	auticko.prezuj_pneu(zima)
	auticko.dej_do_kufru_retezy()	

trabant.jed()
\end{minted}

\captionof{listing}{For}
\label{code:for}
\end{code}
\end{minipage}
\begin{minipage}[t]{.45\textwidth}
\vspace{1.5cm}
ř. 1:	Provede se jednou\\
ř. 3:	Procházíme seznam všech aut (za \uv{in})\\
Všechna auta jsou postupně po jednom uložena do proměnné \uv{auticko}\\
ř. 4-5:	S autem, které je zrovna na řadě se provedou zadané příkazy - postupně se provedou se všemi auty ze seznamu\\
ř. 7:	Po zpracování celého seznamu pokračuje program dále
\end{minipage}\\

\subsubsection{Break}
Jakýkoliv cyklus můžeme také ukončit kdykoliv v jeho těle (mezi příkazy, které jsou v něm napsané). Slouží k tomu příkaz \textbf{\uv{break}}. Ve chvíli kdy \textbf{program} dorazí na řádek s tímto příkazem, \textbf{okamžitě skočí za cyklus}, ve kterém je tento příkaz zapsaný.\\
\begin{minipage}[t]{.45\textwidth}
\begin{code}
\begin{minted}[linenos]{python}
trabant.jed()

while trabant.chci_palivo():
	benzinka.vyber_korunky(trabant)
	if not benzinka.vyber_penez_v_poradku():
		break
	benzinka.dopln_trochu_paliva(trabant)
	
trabant.jed()	
\end{minted}

\captionof{listing}{Break}
\label{code:break}
\end{code}
\end{minipage}
\begin{minipage}[t]{.45\textwidth}
\vspace{3cm}
ř. 1:	Provede se jednou\\
ř. 3:	Kontrola, zda trabant chce palivo\\
ř. 5:	Kontrola, zda trabant zaplatil\\
ř. 6:	Pokud trabant nezaplatil, provede se příkaz \uv{break} - program skočí \textbf{za} cyklus, tedy na řádek 9\\
ř. 9:	Ve chvíli, kdy podmínka splněna není (nechce palivo (False)) pokračuje program dále za celým cyklem (za všemi řádky uvnitř cyklu)
\end{minipage}\\

\subsubsection{Continue}
Pomocí předchozího \uv{break} se ukončil celý průběh cyklu. Můžeme také ukončit průběh cyklu jen v jednom \uv{kolečku} (pro aktuální průchod). Cyklus pak bude pokračovat v dalším kolečku - od jeho prvního řádku.\\

\vspace{1cm}
\textit{Na příkladu níže budou přezuty a budou mít řetězy všechna auta, kromě trabantu.}\\
\begin{minipage}[t]{.45\textwidth}
\begin{code}
\begin{minted}[linenos]{python}
hra.zmen_rocni_obdobi(zima)

for auticko in hra.get_vsechna_auta():
	if auticko is trabant:
		continue
	auticko.prezuj_pneu(zima)
	auticko.dej_do_kufru_retezy()	

trabant.jed()
\end{minted}

\captionof{listing}{Continue}
\label{code:continue}
\end{code}
\end{minipage}
\begin{minipage}[t]{.45\textwidth}
\vspace{1.5cm}
ř. 1:	Provede se jednou\\
ř. 3:	Procházíme seznam všech aut (za \uv{in})\\
ř. 4-7:	S autem, které je zrovna na řadě se provedou zadané příkazy - postupně se provedou se všemi auty ze seznamu\\
ř. 4:	Kontrola, zda je auticko trabant\\
ř. 5:	Pokud auticko je trabant provede se \uv{continue} a cyklus skočí do dalšího kolečka - k dalšímu autu.\\
ř. 9:	Po zpracování celého seznamu pokračuje program dále
\end{minipage}\\
 
 

